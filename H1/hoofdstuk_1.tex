\documentclass[../../master_thesis/4_written_text/main/main.tex]{subfiles}

\begin{document}

\section{Hoofdstuk 1}
We bekijken enkele differentiaalvergelijkingen. De eerste twee vergelijkingen die we beschouwen leerden we oplossen in Analyse II, de resterende differentiaalvergelijkingen zijn nieuw aangebracht in deze cursus (Analyse III). 

De (gewone en gewijzigde) Bessel differentiaalvergelijking komt bijvoorbeeld voor bij het oplossen van de Schrodinger vergelijking onder de veronderstelling van cilindrische symmetrie. We zullen deze functies ook tegenkomen wanneer we partiële differentiaalvergelijking behandelen.


\begin{enumerate}
\item lineaire differentiaalvergelijking \hfill karakteristieke veelterm (zie Analyse II)

\item de vergelijking $M(x,y)dx +  N(x,y)dy = 0$ \hfill
(zie ook Analyse II)

\item Euler differentiaalvergelijking $x^2y''(x) + a_1xy'(x) + a_2y(x) = 0$ \hfill oplossing gaat als volgt:
\begin{itemize}
\item maak tranformatie $\begin{cases} t=ln(x) & x>0\\ t=-x  & x<0 \end{cases}$
\item vergeet niet de inverse tranformatie uit te voeren nadien
\end{itemize}

\item Bessel differentiaalvergelijking van de orde $p$ 
\begin{equation}
x^2y''(x) + xy'(x) + (x^2-p^2)y(x) = 0
\end{equation}
\begin{itemize}
\item Besselfuncties van de eerste en de tweede soort $J_p(x)$ en $Y_p(x)$ respectievelijk.
\end{itemize}

\item gewijzigde Bessel differentiaalvergelijking van orde $p$ (let op het minteken)
\begin{equation}
x^2y''(x) + xy'(x) - (x^2-p^2)y(x) = 0
\end{equation} 
\begin{itemize}
\item gewijzigde Besselfuncties van de eerste en de tweede soort, $I_p(x)$ en $K_p(x)$ respectievelijk.
\end{itemize}

\end{enumerate}



\end{document}
