\documentclass[../../master_thesis/4_written_text/main/main.tex]{subfiles}

\begin{document}

\section{Hoofdstuk 1}
We bekijken enkele differentiaalvergelijkingen. De eerste twee vergelijkingen die we beschouwen leerden we oplossen in Analyse II, de resterende differentiaalvergelijkingen zijn nieuw aangebracht in deze cursus (Analyse III).

\begin{enumerate}
\item lineaire differentiaalvergelijking \hfill karakteristieke veelterm (zie Analyse II)

\item de vergelijking $M(x,y)dx  N(x,y)dy = 0$ \hfill
(zie ook Analyse II)

\item Euler differentiaalvergelijking $x^2y''(x) + a_1xy'(x) + a_2y(x) = 0$ \hfill oplossing gaat als volgt:
\begin{itemize}
\item maak tranformatie $\begin{cases} t=ln(x) & x>0\\ t=-x  & x<0 \end{cases}$
\item vergeet niet de inverse tranformatie uit te voeren nadien
\end{itemize}

\item Bessel differentiaalvergelijking van de orde $p$ namelijk $x^2y''(x) + xy'(x) + (x^2-p^2)y(x) = 0$ 
\begin{itemize}
\item Besselfunctie van de eerste en de tweede soort voldoen aan de vergelijking, $J_p(x)$ en $Y_p(x)$ respectievelijk.
\end{itemize}
Deze differentiaalvergelijking komt bijvoorbeeld voor bij het oplossen van de Schrodinger vergelijking onder de veronderstelling van cilindrische symmetrie. We zullen deze functies ook tegenkomen wanneer we partiële differentiaalvergelijking behandelen.
\end{enumerate}



\end{document}
